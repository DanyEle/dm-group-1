\documentclass[a4paper]{article}

%% Language and font encodings
\usepackage[english]{babel}
\usepackage[utf8x]{inputenc}

\usepackage{booktabs}
\usepackage{tabu}
\usepackage{cancel}


%% Sets page size and margins
\usepackage[a4paper,top=2cm,bottom=0.5cm,left=2cm,right=2cm,marginparwidth=1.75cm]{geometry}

%% Useful packages
\usepackage{amsmath}
\usepackage{enumitem}
\usepackage{listings}
\usepackage{subfigure}

\usepackage{relsize}

\usepackage{pdflscape}
\usepackage{graphicx}
%\usepackage{apacite}
\usepackage[colorinlistoftodos]{todonotes}
\usepackage[colorlinks=true, allcolors=blue]{hyperref}


\title{Data Mining Project To-Do List}



\begin{document}
\begin{landscape}
\renewcommand{\abstractname}{Summary}

\maketitle

\section{Assignment 1 - Data Understanding}
\begin{table}[]
\begin{tabular}{|l|l|l|l|l|l|l|}
\hline
\textbf{\begin{tabular}[c]{@{}l@{}}Task \\ Index\end{tabular}} & \textbf{Task}                                                                                                                                                           & \textbf{Person} & \textbf{Output}                                                                                                                                                                                                                                                                                                    & \textbf{Remarks} & \textbf{Status} & \textbf{Deadline} \\ \hline
1                                                              & \begin{tabular}[c]{@{}l@{}}Determine the quality of the data wrt:\\ - Syntactic Accuracy\\ - Semantic Accuracy\end{tabular}                                             & Maddalena       & \begin{tabular}[c]{@{}l@{}}- Analysis of the syntactic accuracy of all attributes in the data \\ with plots and a textual explanation of the most interesting results\\ \\ - Analysis of the semantic accuracy of all attributes in the data\\ with plots and a textual explanation of the results\end{tabular}    & \textbf{}        & To-Do           & 8/10/2018         \\ \hline
2                                                              & Find outliers via boxplots                                                                                                                                              & Gemma           & \begin{tabular}[c]{@{}l@{}}- Boxplot of the attributes in the dataset to identify outliers\\ Textual explanation of the results obtained via the boxplots\end{tabular}                                                                                                                                             & \textbf{}        & To-Do           & 8/10/2018         \\ \hline
3                                                              & \begin{tabular}[c]{@{}l@{}}Discover new or confirm expected dependencies \\ or correlations between attributes.\end{tabular}                                            & Daniele         & \begin{tabular}[c]{@{}l@{}}- Scatter matrix and scatter plot of the attributes in the dataset and\\ textual explanation of the results obtained via the scatter plots\end{tabular}                                                                                                                                 & \textbf{}        & To-Do           & 8/10/2018         \\ \hline
4                                                              & \begin{tabular}[c]{@{}l@{}}- Detect and examine missing values,\\  possibly hidden by default values.\\ - Check specific application dependent assumptions\end{tabular} & Riccardo        & \begin{tabular}[c]{@{}l@{}}- Analysis to identify missing values in the dataset, with an\\ explanation of the identified missing values and eventually plots\\ thereof\\ - Plot of the distribution of attributes in the dataset\\ {[}QQPlot for identifying the normal distribution of attributes{]}\end{tabular} & \textbf{}        & To-Do           & 8/10/2018         \\ \hline
5                                                              & \begin{tabular}[c]{@{}l@{}}BONUS: Identify possible data normalization \\ to be applied to the data\end{tabular}                                                        &                 & \begin{tabular}[c]{@{}l@{}}Data properly normalized and ready to be fed to data mining\\ algorithms\end{tabular}                                                                                                                                                                                                   &                  & To-Do           &                   \\ \hline
\end{tabular}
\end{table}





\bibliographystyle{plain}
\bibliography{bibtex}




\end{landscape}
\end{document}
              