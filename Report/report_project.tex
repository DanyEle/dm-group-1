\documentclass[a4paper]{article}
%~~~~~~~~~~~~~~~~~~~~~~~~~~~~~Packages~~~~~~~~~~~~~~~~~~~~~~~~~~~~%
\usepackage[english]{babel}
\usepackage[utf8x]{inputenc}
\usepackage{booktabs}
\usepackage{tabu}
\usepackage{cancel}
%% Sets page size and margins
\usepackage[a4paper,top=2cm,bottom=2cm,left=2cm,right=2cm,marginparwidth=1.75cm]{geometry}
\usepackage{amsmath}
\usepackage{enumitem}
\usepackage{listings}
\usepackage{subfigure}
\usepackage{relsize}
\usepackage{graphicx}
%\usepackage{apacite}
\usepackage{pgfplots}
\usepackage{comment}
\usepackage[colorinlistoftodos]{todonotes}
\usepackage[colorlinks=true, allcolors=blue]{hyperref}

%~~~~~~~~~~~~~~~~~~~~~~~~~~~~~Commands~~~~~~~~~~~~~~~~~~~~~~~~~~~~%

\renewcommand{\abstractname}{Summary}

%~~~~~~~~~~~~~~~~~~~~~~~~~~~~~pgfPlots~~~~~~~~~~~~~~~~~~~~~~~~~~~~%

\pgfplotsset{compat=1.8}
\usepgfplotslibrary{statistics}
%%%%%%%%%%%%%%%%%%%%%%%%%%%%%%%%%%%%%%%%%%%%%%%%%%%%%%%%%%%%%%%%%%%%

\begin{document}

\begin{titlepage}
\begin{center}
\vspace{3cm}

\Large

\vspace{2cm}

  \includegraphics[scale=0.3]{imgs/Cherubino.jpg}

\vspace{2.5cm}

{\Huge \sc Data mining project report}

\vspace{1cm}

Maddalena Amendola

\vspace{1cm}

Daniele Gadler

\vspace{1cm}

Riccardo Manetti

\vspace{1cm}

Gemma Martini

\vfill

\today

\end{center}
\end{titlepage}

%%%%%%%%%%%%~~~~~~~~~~~~~~~~~~~~~~~~~~~~~~~~~~~~~~~~~~~%%%%%%%%%%%%

\tableofcontents
\newpage

%%%%%%%%%%%%%%~~~~~~~~~~~~~~~~~~~~~~~~~~~~~~~~~~~~~~~~~~~%%%%%%%%%%
%\author{  Group 1\\ Amendola Maddalena, Daniele Gadler, Gemma Martini, Riccardo Manetti  \\Department of Computer Science, University of Pisa \\ \date{ 1nd Semester of Academic Year 2018--2019}

%\begin{abstract}
 %  This is the abstract text
%\end{abstract}


\section{Data Understanding}

This part contains a first analysis of the data contained in the credit card dataset.

\subsection{BoxPlots by Gemma}
These boxplots are created with Python and give us the chance to understand the shape of the given dataset, in order to find out possible outlayers, which may give ah hint on the many characteristics of the data.
\subsubsection{Age}
From this plot we can observe that
\begin{center}
\includegraphics[width=0.9\textwidth]{../Code/boxPlotsGemma/boxplots/age.png}
\end{center}

\subsubsection{Limit}
\includegraphics[width=0.9\textwidth]{../Code/boxPlotsGemma/boxplots/limit.png}


\subsubsection{ba}
\includegraphics[width=0.9\textwidth]{../Code/boxPlotsGemma/boxplots/ba.png}


\subsubsection{pa}
\includegraphics[width=0.9\textwidth]{../Code/boxPlotsGemma/boxplots/pa.png}


\subsubsection{ps}
\includegraphics[width=0.9\textwidth]{../Code/boxPlotsGemma/boxplots/ps.png}


\bibliographystyle{plain}
\bibliography{bibtex}

\end{document}
              
